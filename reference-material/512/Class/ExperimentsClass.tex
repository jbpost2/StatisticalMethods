\begin{center}\large\textbf{Readings: 7.2 and 7.3, pg 244-255}\\
\normalsize \end{center}
\large \hlinewd{2pt}
~\\
\textbf{Example: } An experiment was run to determine the effects of adding phosphorous ($0, 147, 294, 441$ $kg/m^2$) and nitrogen ($0, 45, 90, 135$ $kg/m^2$) to the soil of a certain type of grass (a Miscanthus species).  The growth of the plant was of interest and at the end of the growing period the plant was dried and the weight recorded with the final measurement being recorded in megagram per hectare ($0.1$ $kg/m^2$).  Four plots of grass were used in total.  Within each plot, each combination of phosphorous and nitrogen was observed.  A partial data table is given here: 
\begin{center}
\begin{tabular}{c|c|c|c}
\hline
Plot	&P	&N&	Dry yield\\
\hline
1&	0&	135&	1.95\\
1&	0&	45&	3.51\\
1&	0&	90&	2.87\\
1&	0&	0&	2.88\\
1&	294	&45&	2.37\\
1&	294	&0&	3.5\\
1&	294	&135&	3.55\\
1&	294&	90&	4.4\\
...&...&...&...\\
\end{tabular}
\end{center}
Let`s identify (if possible) the response, explanatory variable(s), factor(s), level(s), confounding factor(s), treatment(s), number of replicates, and experimental units.  \\

\newpage

Sources of Variation in the responses of an experiment:
\begin{enumerate}
\item \textbf{Treatment effect} - we hope there is an effect due to the variables we control
\item \textbf{Identified confounding variables} - We record some variables that are not of interest, but we think may have an effect on the response.
\item \textbf{Unidentified sources (Experimental Error or error variation)} -
		\begin{enumerate}
			\item Inherent variability in experimental units - Experimental units are different! \\
		Ex: No two people, paper towels, concrete blocks, or even lab rats are exactly the same.\\
		Consequence: Experimental units respond differently to the same treatment
			\item Measurement error - Multiple measurements of a same experimental unit typically contain error.\\
			If the same experimental unit is measured more than once, will the value be the same?\\
			Ex: Blood Pressure, Quality Ratings of food, Break a water sample in two, measure each for bacteria
			\item Variations in applying/creating treatments\\
		The treatment is not clearly defined, leaving room for interpretation.  \\
			Ex:  Two researchers mix concrete, will it come out exactly the same? Ovens don`t heat exactly the same, etc.
			\item Effects from any other extraneous (or lurking) variables - Extraneous variables are those variables that are not part of the treatment, but may influence the response.\\
		\end{enumerate}
\end{enumerate}
Let`s identify these in the previous example.\\

