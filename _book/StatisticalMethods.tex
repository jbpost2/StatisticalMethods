\documentclass[]{book}
\usepackage{lmodern}
\usepackage{amssymb,amsmath}
\usepackage{ifxetex,ifluatex}
\usepackage{fixltx2e} % provides \textsubscript
\ifnum 0\ifxetex 1\fi\ifluatex 1\fi=0 % if pdftex
  \usepackage[T1]{fontenc}
  \usepackage[utf8]{inputenc}
\else % if luatex or xelatex
  \ifxetex
    \usepackage{mathspec}
  \else
    \usepackage{fontspec}
  \fi
  \defaultfontfeatures{Ligatures=TeX,Scale=MatchLowercase}
\fi
% use upquote if available, for straight quotes in verbatim environments
\IfFileExists{upquote.sty}{\usepackage{upquote}}{}
% use microtype if available
\IfFileExists{microtype.sty}{%
\usepackage{microtype}
\UseMicrotypeSet[protrusion]{basicmath} % disable protrusion for tt fonts
}{}
\usepackage[margin=1in]{geometry}
\usepackage{hyperref}
\hypersetup{unicode=true,
            pdftitle={A Minimal Book Example},
            pdfauthor={Yihui Xie},
            pdfborder={0 0 0},
            breaklinks=true}
\urlstyle{same}  % don't use monospace font for urls
\usepackage{natbib}
\bibliographystyle{apalike}
\usepackage{color}
\usepackage{fancyvrb}
\newcommand{\VerbBar}{|}
\newcommand{\VERB}{\Verb[commandchars=\\\{\}]}
\DefineVerbatimEnvironment{Highlighting}{Verbatim}{commandchars=\\\{\}}
% Add ',fontsize=\small' for more characters per line
\usepackage{framed}
\definecolor{shadecolor}{RGB}{248,248,248}
\newenvironment{Shaded}{\begin{snugshade}}{\end{snugshade}}
\newcommand{\AlertTok}[1]{\textcolor[rgb]{0.94,0.16,0.16}{#1}}
\newcommand{\AnnotationTok}[1]{\textcolor[rgb]{0.56,0.35,0.01}{\textbf{\textit{#1}}}}
\newcommand{\AttributeTok}[1]{\textcolor[rgb]{0.77,0.63,0.00}{#1}}
\newcommand{\BaseNTok}[1]{\textcolor[rgb]{0.00,0.00,0.81}{#1}}
\newcommand{\BuiltInTok}[1]{#1}
\newcommand{\CharTok}[1]{\textcolor[rgb]{0.31,0.60,0.02}{#1}}
\newcommand{\CommentTok}[1]{\textcolor[rgb]{0.56,0.35,0.01}{\textit{#1}}}
\newcommand{\CommentVarTok}[1]{\textcolor[rgb]{0.56,0.35,0.01}{\textbf{\textit{#1}}}}
\newcommand{\ConstantTok}[1]{\textcolor[rgb]{0.00,0.00,0.00}{#1}}
\newcommand{\ControlFlowTok}[1]{\textcolor[rgb]{0.13,0.29,0.53}{\textbf{#1}}}
\newcommand{\DataTypeTok}[1]{\textcolor[rgb]{0.13,0.29,0.53}{#1}}
\newcommand{\DecValTok}[1]{\textcolor[rgb]{0.00,0.00,0.81}{#1}}
\newcommand{\DocumentationTok}[1]{\textcolor[rgb]{0.56,0.35,0.01}{\textbf{\textit{#1}}}}
\newcommand{\ErrorTok}[1]{\textcolor[rgb]{0.64,0.00,0.00}{\textbf{#1}}}
\newcommand{\ExtensionTok}[1]{#1}
\newcommand{\FloatTok}[1]{\textcolor[rgb]{0.00,0.00,0.81}{#1}}
\newcommand{\FunctionTok}[1]{\textcolor[rgb]{0.00,0.00,0.00}{#1}}
\newcommand{\ImportTok}[1]{#1}
\newcommand{\InformationTok}[1]{\textcolor[rgb]{0.56,0.35,0.01}{\textbf{\textit{#1}}}}
\newcommand{\KeywordTok}[1]{\textcolor[rgb]{0.13,0.29,0.53}{\textbf{#1}}}
\newcommand{\NormalTok}[1]{#1}
\newcommand{\OperatorTok}[1]{\textcolor[rgb]{0.81,0.36,0.00}{\textbf{#1}}}
\newcommand{\OtherTok}[1]{\textcolor[rgb]{0.56,0.35,0.01}{#1}}
\newcommand{\PreprocessorTok}[1]{\textcolor[rgb]{0.56,0.35,0.01}{\textit{#1}}}
\newcommand{\RegionMarkerTok}[1]{#1}
\newcommand{\SpecialCharTok}[1]{\textcolor[rgb]{0.00,0.00,0.00}{#1}}
\newcommand{\SpecialStringTok}[1]{\textcolor[rgb]{0.31,0.60,0.02}{#1}}
\newcommand{\StringTok}[1]{\textcolor[rgb]{0.31,0.60,0.02}{#1}}
\newcommand{\VariableTok}[1]{\textcolor[rgb]{0.00,0.00,0.00}{#1}}
\newcommand{\VerbatimStringTok}[1]{\textcolor[rgb]{0.31,0.60,0.02}{#1}}
\newcommand{\WarningTok}[1]{\textcolor[rgb]{0.56,0.35,0.01}{\textbf{\textit{#1}}}}
\usepackage{longtable,booktabs}
\usepackage{graphicx,grffile}
\makeatletter
\def\maxwidth{\ifdim\Gin@nat@width>\linewidth\linewidth\else\Gin@nat@width\fi}
\def\maxheight{\ifdim\Gin@nat@height>\textheight\textheight\else\Gin@nat@height\fi}
\makeatother
% Scale images if necessary, so that they will not overflow the page
% margins by default, and it is still possible to overwrite the defaults
% using explicit options in \includegraphics[width, height, ...]{}
\setkeys{Gin}{width=\maxwidth,height=\maxheight,keepaspectratio}
\IfFileExists{parskip.sty}{%
\usepackage{parskip}
}{% else
\setlength{\parindent}{0pt}
\setlength{\parskip}{6pt plus 2pt minus 1pt}
}
\setlength{\emergencystretch}{3em}  % prevent overfull lines
\providecommand{\tightlist}{%
  \setlength{\itemsep}{0pt}\setlength{\parskip}{0pt}}
\setcounter{secnumdepth}{5}
% Redefines (sub)paragraphs to behave more like sections
\ifx\paragraph\undefined\else
\let\oldparagraph\paragraph
\renewcommand{\paragraph}[1]{\oldparagraph{#1}\mbox{}}
\fi
\ifx\subparagraph\undefined\else
\let\oldsubparagraph\subparagraph
\renewcommand{\subparagraph}[1]{\oldsubparagraph{#1}\mbox{}}
\fi

%%% Use protect on footnotes to avoid problems with footnotes in titles
\let\rmarkdownfootnote\footnote%
\def\footnote{\protect\rmarkdownfootnote}

%%% Change title format to be more compact
\usepackage{titling}

% Create subtitle command for use in maketitle
\newcommand{\subtitle}[1]{
  \posttitle{
    \begin{center}\large#1\end{center}
    }
}

\setlength{\droptitle}{-2em}

  \title{A Minimal Book Example}
    \pretitle{\vspace{\droptitle}\centering\huge}
  \posttitle{\par}
    \author{Yihui Xie}
    \preauthor{\centering\large\emph}
  \postauthor{\par}
      \predate{\centering\large\emph}
  \postdate{\par}
    \date{2019-10-12}

\usepackage{booktabs}

\begin{document}
\maketitle

{
\setcounter{tocdepth}{1}
\tableofcontents
}
\hypertarget{prerequisites}{%
\chapter{Prerequisites}\label{prerequisites}}

This is a \emph{sample} book written in \textbf{Markdown}. You can use anything that Pandoc's Markdown supports, e.g., a math equation \(a^2 + b^2 = c^2\). A

The \textbf{bookdown} package can be installed from CRAN or Github:

\begin{Shaded}
\begin{Highlighting}[]
\KeywordTok{install.packages}\NormalTok{(}\StringTok{"bookdown"}\NormalTok{)}
\CommentTok{# or the development version}
\CommentTok{# devtools::install_github("rstudio/bookdown")}
\end{Highlighting}
\end{Shaded}

Remember each Rmd file contains one and only one chapter, and a chapter is defined by the first-level heading \texttt{\#}.

To compile this example to PDF, you need XeLaTeX. You are recommended to install TinyTeX (which includes XeLaTeX): \url{https://yihui.name/tinytex/}.

\hypertarget{sampling-experiments-and-exploratory-data-analysis}{%
\chapter{Sampling, Experiments, and Exploratory Data Analysis}\label{sampling-experiments-and-exploratory-data-analysis}}

\hypertarget{data-in-the-wild}{%
\section{Data in the Wild}\label{data-in-the-wild}}

Words

\hypertarget{experiment-background}{%
\subsection{Experiment Background}\label{experiment-background}}

Marketing example. Goal to describe the customers, how they tend to purchase/shop, and maybe find some shared qualities in order to adverstise curated packages to folks.

Define basic things like population, parameters, statistics, and sample.

Discuss conceptual vs actual populations and when we might care about one or the other. Our ``sample'' is really a bit of data from the conceptual population. Or we could consider it as the population and we just want to describe it.

\hypertarget{selecting-response-variables}{%
\subsection{Selecting Response Variables}\label{selecting-response-variables}}

Marketing example with data such as Clicks, Impressions, Total Revenue, Total Spent, Average Order Value, Sport, Time of visit/purchase, Campaigns running, etc.

\hypertarget{identifying-sources-of-variation}{%
\subsection{Identifying Sources of Variation}\label{identifying-sources-of-variation}}

Consider variables linked to the user. Age, other accounts, etc.

\hypertarget{choose-an-experimental-design}{%
\subsection{Choose an Experimental Design }\label{choose-an-experimental-design}}

Discuss our ``sampling'' scheme vs a random sample. This seems like a case where we aren't doing a ``good'' scheme but not much else could be done\ldots{}

Maybe talk about how in the future you could do alternate email ads or something and do an AB type study.

\hypertarget{peform-the-test}{%
\subsection{Peform the Test }\label{peform-the-test}}

Get the data from google analytics or whatever, have a plan for updating each month?

\hypertarget{look-at-the-data}{%
\subsection{Look at the Data }\label{look-at-the-data}}

Careful discussion of not selecting a modeling technique based on this unless it is a pilot study or an exploratory study else we have increased our nominal type I error rate\ldots{}

(sometimes EDA sometimes data validation only/cleaning - more formal experiments)

Spend a lot of time here talking about graphs of different types. Sample means, sample variances, etc.

Discuss population curves vs sample histograms and the relationship.

\hypertarget{statistically-analyze-the-data}{%
\subsection{Statistically Analyze the Data}\label{statistically-analyze-the-data}}

New variables as functions of old?

Not a formal test here but comparisons of interest etc.

\hypertarget{draw-conclusions}{%
\subsection{Draw conclusions}\label{draw-conclusions}}

What actionable things have we found? Likely some trends to investigate further. Perhaps run an experiment to formally see if some alteration can be effective.

What can we conclude realistically from this data? To what population are we talking?

\hypertarget{statistical-testing-ideas}{%
\section{Statistical Testing Ideas}\label{statistical-testing-ideas}}

\hypertarget{experiment-background-1}{%
\subsection{Experiment Background}\label{experiment-background-1}}

This example would lend itself to a reasonably easy randomization test or simulation based test. Maybe an AB type study where we swap labels and do that with a nice visual.

Maybe third example with simulation test.

\hypertarget{selecting-response-variables-1}{%
\subsection{Selecting Response Variables}\label{selecting-response-variables-1}}

\hypertarget{identifying-sources-of-variation-1}{%
\subsection{Identifying Sources of Variation}\label{identifying-sources-of-variation-1}}

\hypertarget{choose-an-experimental-design-1}{%
\subsection{Choose an Experimental Design}\label{choose-an-experimental-design-1}}

Good discussion of what makes a good sampling design. Maybe a statified example like the river and selecting houses example as a quick expose of the issues with not doing a truly random sampling technique.

Basics of experimental design (randomization, replication, error control ideas).

Recap benefits of doing an experiment vs an observational study.

\hypertarget{peform-the-test-1}{%
\subsection{Peform the Test}\label{peform-the-test-1}}

\hypertarget{explore-the-data}{%
\subsection{Explore the Data}\label{explore-the-data}}

NHST paradigm with false discovery?

\hypertarget{statistically-analyze-the-data-1}{%
\subsection{Statistically Analyze the Data}\label{statistically-analyze-the-data-1}}

\hypertarget{draw-conclusions-1}{%
\subsection{Draw conclusions}\label{draw-conclusions-1}}

\hypertarget{point-estimates}{%
\chapter{Point Estimates}\label{point-estimates}}

Learning objectives for this lesson:
- How to estimate a mean
- Definition of ``convenience sample''
- Definition of ``systematic sample''
- Benefits/drawbacks to both approaches
- Understand how to estimate a mean
- Understand how to estimate a quantile
- Understand implicit assumptions for these approaches

\hypertarget{estiamte-with-means}{%
\section{Estiamte with means}\label{estiamte-with-means}}

\hypertarget{experiment-background-2}{%
\subsection{Experiment background}\label{experiment-background-2}}

Someone wants to know how much of something they need to satisfy some population
To get a good estimate of this, we can use the average amount for each one and then multiply by the whole population

\hypertarget{estimate-with-quantiles}{%
\section{Estimate with quantiles}\label{estimate-with-quantiles}}

\hypertarget{experiment-background-3}{%
\subsection{Experiment background}\label{experiment-background-3}}

Big Deborah's is making new packaging for their cookies. The engineer responsible for the new desing needs to make sure that the packaging fits the new cookies. While the cookie manufacturing process is standardized, there's inevitably some degree of variation in cookie size. After discussing the issue with corporate, the engineer decides that a the new cookie sleeves should be large enough to fit 95\% of cookies that are baked. (The largest five percent will be marketed separately as ``JUMBO'' cookies.)

\hypertarget{define-the-object-of-the-experiment}{%
\subsection{Define the object of the experiment}\label{define-the-object-of-the-experiment}}

The Engineer is tasked with determining how large the cooke sleeve needs to be. There's no way for her to know the size of every cookie that Big Deborah's has made (or will make going forward!), so she'll need to collect data on existing cookies to inform her cookie sleeve size determination.

\hypertarget{select-appropriate-response-variables}{%
\subsection{Select appropriate response variables}\label{select-appropriate-response-variables}}

If the maximum distance from any one point on the (round) cookie's perimeter to any other point is smaller than the diameter of the cookie sleeve, then the cookie will fit. This makes ``cookie diameter'' a good measure for this test. It is easy to measure for each cookie and is directly relevant to the experiment's objective.

{[}probably have something in here about {]}

\hypertarget{identify-sources-of-variation}{%
\subsection{IDentify sources of variation}\label{identify-sources-of-variation}}

While the manufacturing process is standardized, there is variation in size from one cookie to the next. This is one source of variation. The engineer isn't sure of any others. However, she knows that cookies are made in multiple factories, and that each factory has multiple ovens. Ovens and factories could also be sources of variation.

\hypertarget{choose-an-experimental-design-2}{%
\subsection{Choose an experimental design}\label{choose-an-experimental-design-2}}

The Engineer knows that she needs to look at multiple cookies, since she knows that there is variation in diameter from one cookie to the next. One option would be to just use the remaining cookies in the box she has in her office (22 of the 25-count box remain). {[}something about convenience sample{]} However, she knows that cookies from the same oven are typically packaged together. If there is variation from one oven to the next, looking at the cookies she has in her office may not tell the whole story.

Instead, she chooses to take every 20th cookie manufactured off the assembly line until she gets 500 cookies. {[}something about systematic sample{]}

\hypertarget{perform-the-test}{%
\subsection{Perform the test}\label{perform-the-test}}

The day of the test comes, and the Engineer starts collecting cookies. However, problems arise! The plan has to shut down half-way through, so she only gets 431 cookies instead of the 500 she thought she would. However, she measures the diameters of each cookie and records the data in a spreadsheet.

\hypertarget{statistically-analyze-the-data-2}{%
\subsection{Statistically analyze the data}\label{statistically-analyze-the-data-2}}

The initial plan had been to rank-order the 500 cookies and estimate the 95th percentile using the diamter of the 475th largets cookie. Since we didn't get all of our data, we have to improvise. 431 doesn't neatly yield a value such that exactly 95\% are less than or equal and 5\% are greater than or equal. One option is to choose the 410th largest cookie to estimate our percentile. Slightly more than 95\% of cookies will have smaller diameters than this. Alternatively, we could interpolate between the 409th and 410th cookies. {[}reasons and logic and math for each of these{]}

\hypertarget{draw-conclusions-2}{%
\subsection{Draw conclusions}\label{draw-conclusions-2}}

Based on this study, the Engineer concludes that a cookie sleeve large enough for a cookie of diameter XX will be big enough to contain 95\% of Big Deborah cookies.

\hypertarget{discussion}{%
\subsection{Discussion}\label{discussion}}

\begin{itemize}
\tightlist
\item
  pros and cons to the approach chosen
\item
  generalizing to other types of point estimates
\end{itemize}

\hypertarget{accounting-for-uncertainty}{%
\chapter{Accounting for Uncertainty}\label{accounting-for-uncertainty}}

Some \emph{significant} applications are demonstrated in this chapter.

\hypertarget{example-one}{%
\section{Example one}\label{example-one}}

\hypertarget{example-two}{%
\section{Example two}\label{example-two}}

\hypertarget{HT}{%
\chapter{Inference via Hypothesis Tests for One Sample}\label{HT}}

We have finished a nice book.

\hypertarget{CI}{%
\chapter{Inference via Confidence Intervals for One Sample}\label{CI}}

We have finished a nice book.

\hypertarget{twocategorical}{%
\chapter{Inference for Two Categorical Variables}\label{twocategorical}}

We have finished a nice book.

\hypertarget{anova}{%
\chapter{One-Way ANOVA}\label{anova}}

We have finished a nice book.

\hypertarget{multiway}{%
\chapter{Multi-way ANOVA}\label{multiway}}

We have finished a nice book.

\hypertarget{block}{%
\chapter{Block Designs}\label{block}}

We have finished a nice book.

\hypertarget{regression}{%
\chapter{Regression Models}\label{regression}}

We have finished a nice book.

\hypertarget{glm}{%
\chapter{The General Linear Model}\label{glm}}

We have finished a nice book.

\hypertarget{mixedmodels}{%
\chapter{Mixed Models}\label{mixedmodels}}

We have finished a nice book.

\hypertarget{repeatedmeasures}{%
\chapter{Split Plot and Repeated Measures Designs}\label{repeatedmeasures}}

We have finished a nice book.

\hypertarget{logistic}{%
\chapter{Logistic Regression and Generalized Linear Models}\label{logistic}}

\hypertarget{stuff-here}{%
\section{Stuff here}\label{stuff-here}}

We have finished a nice book.

\hypertarget{glmm}{%
\chapter{Generalized Linear Mixed Models}\label{glmm}}

We have finished a nice book.

\hypertarget{learningobj}{%
\chapter{Appendix - Learning Objectives}\label{learningobj}}

\hypertarget{book-level}{%
\section{Book-level}\label{book-level}}

After reading this book you will be able to:

\begin{itemize}
\tightlist
\item
  identify relevent sources of variability for a potential study

  \begin{itemize}
  \tightlist
  \item
    covariates
  \item
    noise variables
  \item
    random effects
  \end{itemize}
\item
  utilize principles of design to plan a reasonable experiment to help answer questions of interest

  \begin{itemize}
  \tightlist
  \item
    randomization
  \item
    systematic variation of factors/covariates
  \item
    factor identifiability
  \end{itemize}
\item
  compare and contrast methods for designing an experiment when the goal of a study is prediction versus when the goal is statistical inference

  \begin{itemize}
  \item
  \end{itemize}
\item
  articulate the scope of inferential conclusions in light of the method of data collection, the experimental design used, and the statistical analysis applied

  \begin{itemize}
  \tightlist
  \item
    limitations due to sampling/sample frame
  \item
    missing data
  \item
    modeling assumptions
  \item
    sampling assumptions
  \item
    requirements for causal inference
  \end{itemize}
\item
  choose appropriate numerical summaries and graphical displays for a set of data and create these using software

  \begin{itemize}
  \tightlist
  \item
    when to use tables vs.~a picture
  \item
    types of graphical displays

    \begin{itemize}
    \tightlist
    \item
      bar charts
    \item
      pie charts
    \item
      plotting data vice just predictions/conclusions
    \item
      when to include uncertainty bounds
    \item
      five-number summaries
    \item
      means vs.~medians
    \item
      general plotting recommendations
    \item
      use of colors in you plots (discrete vs.~divergent vs.~continuous color scales, gray-scale, color-blind-friendly scales)
    \end{itemize}
  \item
    use of annotations
  \item
    general graphical design philosophy (building a chart to illustrate a conclusion)
  \item
    trade-offs between detail and interpretability
  \item
    not screwing up your axes
  \end{itemize}
\item
  explain the general concept of point estimation and how to account for sampling variability

  \begin{itemize}
  \tightlist
  \item
    definition
  \item
    identify the right point estimate for your response variable of interest
  \item
    estimating uncertainty for point estimates

    \begin{itemize}
    \tightlist
    \item
      normal approximation
    \item
      bootstrap CI
    \item
      others?
    \end{itemize}
  \end{itemize}
\item
  explain the importance of statistical distributions when conducting statistical inference

  \begin{itemize}
  \tightlist
  \item
    normal distribution and approximations plus properties

    \begin{itemize}
    \tightlist
    \item
      robustness
    \item
      generality
    \item
      CLT
    \end{itemize}
  \item
    costs and benefits of using nonparametric approaches
  \end{itemize}
\item
  describe the fundamental inferential techniques of hypothesis testing and confidence intervals

  \begin{itemize}
  \tightlist
  \item
    identify a null and alternative for a given problem
  \item
    interpret hypotheses
  \item
    characterize the test statistic under the null
  \item
    explain what a rejection region and be able to identify one
  \item
    ????
  \end{itemize}
\item
  compare and contrast the use of and interpretations from hypothesis tests and confidence intervals

  \begin{itemize}
  \tightlist
  \item
    identify when using a CI and NSHT will result in the same conclusion
  \item
    explain when you can use a confidence interval to test for differences (e.g., comparing a single point estimate to a threshold) and when you can't (e.g., when you have CIs for two different means)
  \end{itemize}
\item
  fit statistical models in software and interpret their output

  \begin{itemize}
  \tightlist
  \item
    Which PROCs from SAS? REG, GLM, MIXED, GLIMMIX, others??
  \item
    \texttt{lm()}, \texttt{glm()}, \texttt{anova()} \ldots{}. \texttt{broom}? \texttt{modlr}? \texttt{ciTools}?
  \item
    p-values, point estimates, standard errors, f-statistics, chi-square-statistics, degrees of freedom, SS/MS, residual plots
  \end{itemize}
\item
  Something about understanding the relationships between the models (linear model framework)

  \begin{itemize}
  \tightlist
  \item
    Write statistical models using matrix representaiton
  \item
    identify models written in matrix representation with their representation in software
  \item
    identify when models written in different notation are the same or different
  \item
    describe when specific models will give you the same results

    \begin{itemize}
    \tightlist
    \item
      ANOVA w/ 2 factors and a t-test or a SLR
    \item
      ANCOVA and MLR
    \item
      random effects vs.~fixed effects
    \item
      split plots vs.~more general mixed models
    \item
      logistic regression w/ categorical factors vice contingency table analysis
    \end{itemize}
  \item
    discuss differences in assumptions associated with ANOVA vice SLR/MLR
  \end{itemize}
\item
  Maybe another bit about data types you see, how the assumptions we make impact things
\end{itemize}

\hypertarget{topic-level}{%
\section{Topic-level}\label{topic-level}}

\hypertarget{chapter-2---sampling-design-and-exploratory-data-analysis}{%
\subsection{Chapter 2 - Sampling, Design, and Exploratory Data Analysis}\label{chapter-2---sampling-design-and-exploratory-data-analysis}}

\hypertarget{chapter-3---point-estimation}{%
\subsection{Chapter 3 - Point Estimation}\label{chapter-3---point-estimation}}

\hypertarget{chapter-4---accounting-for-uncertainty-in-estimation}{%
\subsection{Chapter 4 - Accounting for Uncertainty in Estimation}\label{chapter-4---accounting-for-uncertainty-in-estimation}}

\hypertarget{chapter-5---inference-via-hypothesis-testing-for-a-proportion-or-mean}{%
\subsection{Chapter 5 - Inference via Hypothesis Testing for a Proportion or Mean}\label{chapter-5---inference-via-hypothesis-testing-for-a-proportion-or-mean}}

\hypertarget{chapter-6---inference-via-confidence-intervals-for-a-proportion-or-mean}{%
\subsection{Chapter 6 - Inference via Confidence Intervals for a Proportion or Mean}\label{chapter-6---inference-via-confidence-intervals-for-a-proportion-or-mean}}

\hypertarget{chapter-7---inference-on-two-categorical-variables}{%
\subsection{Chapter 7 - Inference on Two Categorical Variables}\label{chapter-7---inference-on-two-categorical-variables}}

\hypertarget{chapter-8---inference-for-multiple-means}{%
\subsection{Chapter 8 - Inference for Multiple Means}\label{chapter-8---inference-for-multiple-means}}

\hypertarget{chapter-9---multiway-anova}{%
\subsection{Chapter 9 - Multiway ANOVA}\label{chapter-9---multiway-anova}}

\hypertarget{chapter-10---block-designs}{%
\subsection{Chapter 10 - Block Designs}\label{chapter-10---block-designs}}

\hypertarget{chapter-11---regression}{%
\subsection{Chapter 11 - Regression}\label{chapter-11---regression}}

\hypertarget{chapter-12---the-general-linear-model}{%
\subsection{Chapter 12 - The General Linear Model}\label{chapter-12---the-general-linear-model}}

\hypertarget{chapter-13---mixed-models}{%
\subsection{Chapter 13 - Mixed Models}\label{chapter-13---mixed-models}}

\hypertarget{chapter-14---repeated-measures-and-split-plot-designs}{%
\subsection{Chapter 14 - Repeated Measures and Split Plot Designs}\label{chapter-14---repeated-measures-and-split-plot-designs}}

\hypertarget{chapter-15---logistic-regression-and-generalized-linear-models}{%
\subsection{Chapter 15 - Logistic Regression and Generalized Linear Models}\label{chapter-15---logistic-regression-and-generalized-linear-models}}

\hypertarget{chapter-16---generalized-linear-mixed-models}{%
\subsection{Chapter 16 - Generalized Linear Mixed Models}\label{chapter-16---generalized-linear-mixed-models}}

\hypertarget{from-st512}{%
\section{From ST512}\label{from-st512}}

WE NEED TO ORGANIZE THESE UNDER DIFFERENT CHAPTERS AT SOME POINT
Learning Objectives

\begin{enumerate}
\def\labelenumi{\arabic{enumi}.}
\item
  Recognize a completely randomized design with one treatment factor and write the corresponding one-way analysis of variance model, with assumptions
\item
  Estimate treatment means
\item
  Estimate the variance among replicates within a treatment
\item
  Construct the analysis of variance table for a one factor analysis of variance, including computing degrees of freedom, sums of squares, mean squares, and F-ratios
\item
  Interpret results and draw conclusions from a one-factor analysis of variance
\item
  Estimate differences between two treatment means in a one factor analysis of variance
\item
  Test differences between two treatment means in a one factor analysis of variance
\item
  Construct a contrast to estimate or test a linear combination of treatment means
\item
  Estimate the standard error of a linear combination of treatment means
\item
  Make inferences about linear combinations of treatment means, including contrasts.
\item
  Obtain and understand SAS output for linear combinations of treatment means, including contrasts.
\item
  Explain when and why corrections for multiple comparisons are needed
\item
  Know when and how to use Tukey's correction for all pairwise comparisons
\item
  Compute Bonferroni confidence intervals
\item
  Create and interpret orthogonal contrasts.
\item
  Define main effects and interactions
\item
  Write contrasts to estimate main effects and interactions
\item
  Estimate these contrasts and their standard errors
\item
  Compute sums of squares associated with these contrasts
\item
  Test hypotheses about the main effects and interactions.
\item
  Identify and define simple effects.
\item
  Identify and define interaction effects.
\item
  Identify and define main effects.
\item
  Understand when to use simple, interaction, and main effects when drawing inferences in a two-way ANOVA.
\item
  Write the analysis of variance model and SAS code for a completely randomized design with two factors
\item
  Test hypotheses and interpret the analysis of variance for a factorial experiment.
\item
  Explain the appropriate use of correlations and compute the correlation coefficient
\item
  Read and interpret a scatterplot and guess the correlation coefficient by examination of a scatter plot
\item
  Interpret the strength and direction of association indicated by the correlation coefficient and judge when a correlation coefficient provides an appropriate summary of a bivariate relationship
\item
  Test the hypothesis that the correlation coefficient is zero using either a t-test or the Fisher z transformation, Compute confidence intervals using Fisher's z transformation
\item
  Write a statistical model for a straight line regression or a multiple regression and explain what all the terms of the model represent
\item
  Explain the assumptions underlying regression models, evaluate whether the assumptions are met
\item
  Estimate the intercept, slope and variance for a simple linear regression model
\item
  Fit a multiple regression model in SAS and interpret the output, use the coefficient of determination to evaluate model fit
\item
  Use a regression model to predict Y for new values of X
\item
  Estimate the variance and standard error of parameters in regression models, test hypotheses about the parameters, and construct confidence intervals for the parameters.
\item
  Explain the difference between a confidence interval and a prediction interval and know when to use each of them
\item
  Construct a confidence interval for the expected value of Y at a given value of X
\item
  Construct a prediction interval for a new value of Y at a given value of X
\item
  Write a linear model in matrix notation
\item
  Find the expectation and variance of a linear combination of random variables, a'Y
\item
  Set up the expressions to calculate parameter estimates and predicted values using the matrix form of the model
\item
  Estimate standard errors for parameter estimates and predicted values
\item
  Use extra sums of squares to test hypotheses about subsets of parameters
\item
  Construct indicator variables for including categorical regressor variables in a linear model
\item
  Understand how to interpret parameters of a general linear model with indicator variables
\item
  Estimate contrasts of treatment means and their standard errors using the general linear model notation and matrix form of the model
\item
  Compare nested models with a lack of fit test to select a model
\item
  Explain what a covariate is and how they are used
\item
  Explain the assumptions of the analysis of covariance model and determine when these assumptions are met
\item
  Fit an analysis of covariance model in SAS and conduct appropriate tests for treatment effects
\item
  Estimate and interpret treatment means and their standard errors adjusted for covariates using SAS, Construct confidence intervals for adjusted treatment means
\item
  Construct and estimate contrasts of treatment means adjusted for covariates and estimate the standard errors and confidence intervals of such contrasts.
\end{enumerate}

Analysis of variance and design of experiments
Recognize each of the following types of experimental designs and determine when each type would be advantageous.
1. completely randomized design
2. randomized complete block design
3. split plot design
Recognize whether factors should be considered fixed effects or random effects and explain the scope of inference for each case.
Recognize whether factors are crossed or nested.
For all of the designs listed and for experiments with crossed and/or nested fixed factors, random factors, or a combination of fixed and random effects, be able to
1. Write the corresponding analysis of variance model, with assumptions, and define all terms
2. Estimate treatment means and their standard errors
3. Construct the analysis of variance table, including computing degrees of freedom, sums of squares, mean squares, and F-ratios
4. Determine whether the assumptions of the model are satisfied
5. Interpret results and draw conclusions
6. Construct and estimate linear combinations of treatment means and their standard errors
7. Test hypotheses and construct confidence intervals about linear combinations of treatment means
8. Explain when and why corrections for multiple comparisons are needed, know when and how to use Tukey's correction for all pairwise comparisons, compute Bonferroni confidence intervals
9. Create and interpret orthogonal contrasts.
10. Define and interpret main effects, simple effects and interactions
11. Use a table of expected mean squares to estimate variance components and determine appropriate F-statistics for testing effects in the analysis of variance
12. Interpret variance components and estimate and interpret the intraclass correlation coefficient.
Regression and correlation
Explain the appropriate use of correlations and compute the correlation coefficient, read and interpret a scatterplot and guess the correlation coefficient by examination of a scatter plot, test the hypothesis that the correlation coefficient is zero using either a t-test or the Fisher z transformation, compute confidence intervals using Fisher's z transformation
You should be able to do the following for fitting models to describe the relationships of one or several variables to a response variable. The regressor variables may be continuous or categorical or a mix of the two (e.g., analysis of covariance models)
1. Write a general linear model, including assumptions, in standard or matrix notation, and explain what all the terms and assumptions represent. Be able to handle models that contain interaction terms, polynomial terms, and dummy variables.
2. Evaluate whether the model assumptions are met
3. Fit a general linear model in SAS and interpret the output
4. Work with the general linear model in matrix form, including finding the expectation and variance of a linear combination of regression coefficients or treatment means
5. Test hypotheses and construct confidence intervals for linear combinations of the parameters
6. Construct and interpret a confidence interval for the expected value of Y at a given value of X
7. Construct and interpret a prediction interval for a new value of Y at a given value of X
8. Use extra sums of squares to test hypotheses about subsets of parameters.
9. Explain what a covariate is and how covariates are used

\hypertarget{for-point-estimates-chapter}{%
\section{For Point Estimates Chapter}\label{for-point-estimates-chapter}}

\begin{itemize}
\item
  Definitions for Mean, Median, Quantile, Percentile
\item
  Explain uses for the above
\item
  Identify the correct point estimate to use for a given test
\item
  Define Systematic Random Sample and Convenience Sample
\item
  Explain strengths and weaknesses of each
\item
  Identify conditions when Systematic and Convenience Sampling may not provide representitive samples
\end{itemize}

\hypertarget{notation}{%
\chapter{Appendix - Notation}\label{notation}}

\hypertarget{standard-notation}{%
\section{Standard notation}\label{standard-notation}}

Vectors of variables are denoted with Roman letters, such as \(x\) and \(Y\). Capital letters denote random variables while lower case letters denote fixed variables. Note that these vectors may be of length 1 depending on context. Bolded values (\textbf{\(x\)}) denote matrices, and in the case of \textbf{\(Y\)}, possibly single-column matrices.

Unknown parameters are denoted with Greek letters, with boldface font indicating matrices.

In most models, \(Y\) will denote the univariate response, \textbf{\(x\)} will describe a matrix of predictor variables, and \(E\) a vector of random errors. The Greek letter \(\beta\) will be commonly used for regression parameters (either with subscripts for each values as in \(\beta_0 + \beta_1 X_1\) or as a vector (as in \(X\beta\)). The letters \(i, j, k,\) and \(l\) will be most commonly used as subscripts or indices. \(N\) will typically denote a sample size (not a random vector), with subscripted versions (\(n_i\)) describing the number of observations in a group, and \(p\) describing the number of parameters in a model beyond the intercept.

We may therefore describe a simple linear regresion model as:

\[Y = x\beta + E\]

In this model, \(Y\) is a \(N\times 1\) random vector, \textbf{\(x\)} is a \(N\times (p + 1)\) matrix of fixed values, and \(E\) is a \(N \times 1\) vector.

\(\pi\) is typically used to describe probability parameters, as in Bernoulli or binomial random variables.

\hypertarget{mixed-models}{%
\section{Mixed models}\label{mixed-models}}

Still need to add something for this

\hypertarget{effects-model-representation}{%
\section{Effects model representation}\label{effects-model-representation}}

In the effects formulation of ANOVA models, additional greek letters (\(\alpha\), \(\gamma\), etc.) will appear as parameter effects, as will \(\mu\), which will typically represent the grand mean. Group-specfic means will be denoted via subscripts: \(\mu_{ij}\). When using this representation, it is convenient to describe a single observation as \(Y_{ijk}\), which is the \(k\)th observation from the group with with the \(i\)th level of the first factor and the \(j\)th level of the second factor. In the main effects version of this model, we have:

\[Y_{ijk} = \mu + \alpha_i + \gamma_j + E_{ijk}\]

We can therefore estimate \(\mu_{ij}\) as \(\hat \mu_{ij} = \frac{1}{n}\sum_{k = 1}^n Y_{ijk} = \bar{Y}_{ij\cdot}\). This ``dot'' notation can be extended to any subscript and indicates summing over the index that has been replaced by the dot. Further note that the ``hat'' over a paremeter value denotes the estimator for that parameter value, and the ``bar'' indicates an average. These features are used generally throughout this book.

\hypertarget{estimators-vs.-estimates}{%
\section{Estimators vs.~Estimates}\label{estimators-vs.-estimates}}

If we want to get pedantic, we can differentiate between estimates and estimators in our notation. Estimators are functions of random variables used to estimate parameters. Estimates are realized values of estimators. To differentiate these, we use Roman letters with hats to represent estimators (\(\hat B = (x'x )^{-1}x'Y\)) and Greek letters with hats to represent estimates (\(\hat \beta = 1.52\)).

\bibliography{book.bib,packages.bib}


\end{document}
